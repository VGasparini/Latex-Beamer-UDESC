
\documentclass{beamer}

\usepackage{graphicx,hyperref,udesc,url}
\usepackage[latin1]{inputenc}
%\usepackage[T1]{fontenc}
\usepackage{booktabs}
\usepackage[portuges]{babel}


\title[Uso de Coq para verifica��o de propriedades de sistemas de tipos]{Uso de Coq para verifica��o de propriedades de sistemas de tipos}

\author[Rafael Castro]{
    Rafael Castro\\\medskip
    {\small \url{rafaelcgs10@gmail.com}}}

\date{22 de Junho de 2018}

    \institute[UDESC]{
        Departamento de Ci\^encia da Computa\c{c}\~ao \\
            Centro de Ci\^encias e Tecnol\'ogicas\\
            Universidade do Estado de Santa Catarina}

\begin{document}

\begin{frame}
\titlepage

\end{frame}

\section{Introdu��o}
\begin{frame}
\frametitle{O que s�o assistentes de provas?}
\begin{itemize}
    \item Assistentes de provas ou provadores interativos s�o programas para o desenvolvimento de provas
          formais.
    \item O n�cleo de um assistente de provas � um verificador, que verifica a consist�ncia l�gica da prova.
    \item Fornecem de maneira interativa de visualizar as informa��es sobre o estado atual da prova.
    \item A verifica��o humana de provas � demorada e sujeita a falhas: �ltimo Teorema de Fermat.

      Assistentes de provas permitem provar coisas que n�o seriam realiz�veis somente com papel e caneta!
\end{itemize}
\end{frame}

\section{Coq}
\begin{frame}
\frametitle{O que � Coq?}
\begin{itemize}
    \item Coq � um assistentes de provas desenvolvido desde 1984 pelo French Institute for Research in Computer Science and Automation (INRIA).
    \item Coq � fruto de sistemas de tipos: baseado em \textbf{Higher order dependently typed polymorphic lambda calculus}, o nomeado Calculus of Constructions (CoC).
    \item Coq suporta programa��o com tipos dependentes.
    \item Coq tem várias linguagens:
        \begin{enumerate}
            \item Gallina: linguagem de programa��o funcional total e com tipos dependentes.
            \item Vernecular: linguagem de comandos. Enunciar teoremas, fun��es...
            \item Tatic: linguagem das t�ticas usada para criar provas com a ajuda do essistente/verificador.
            \item LTac: cria��o de novas t�ticas e procedimentos de prova.
        \end{enumerate}
\end{itemize}
\end{frame}

\section{A linguagem}

\begin{frame}[fragile]
\frametitle{A linguagem: STLC}
\begin{itemize}
  \item C�lculo Lambda Simplesmente Tipado: \textit{Simple Typed Lambda Calculus} (STLC).
\end{itemize}
\begin{verbatim}
t ::= x                       variable
    | \x:T1.t2                abstraction
    | t1 t2                   application
    | true                    constant true
    | false                   constant false
    | if t1 then t2 else t3   conditional

T ::= Bool
    | T1 -> T2
\end{verbatim}
\end{frame}

\begin{frame}[fragile]
\frametitle{Formaliza��o de STLC em Coq}
\begin{verbatim}
Inductive ty : Type :=
  | TBool  : ty
  | TArrow : ty -> ty -> ty.

Inductive tm : Type :=
  | tvar : string -> tm
  | tapp : tm -> tm -> tm
  | tabs : string -> ty -> tm -> tm
  | ttrue : tm
  | tfalse : tm
  | tif : tm -> tm -> tm -> tm.
\end{verbatim}
\end{frame}

\end{document}