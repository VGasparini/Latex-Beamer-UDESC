\documentclass{beamer}

\usepackage{graphicx,hyperref,udesc,url}
\usepackage[latin1]{inputenc}
\usepackage{bussproofs}
%\usepackage[T1]{fontenc}
\usepackage{booktabs}
\usepackage[portuges]{babel}


\title[Extraction of Programs from Proofs]{Extraction of Programs from Proofs}

\author[Rafael Castro]{
    Rafael Castro\\\medskip
    {\small \url{rafaelcgs10@gmail.com}}}

    \institute[UDESC]{
        Departamento de Ci\^encia da Computa\c{c}\~ao \\
            Centro de Ci\^encias e Tecnol\'ogicas\\
            Universidade do Estado de Santa Catarina}

\date{07/03/2018}

\begin{document}

\begin{frame}
\titlepage

\end{frame}

\section{Isomorfismo de Curry-Howard}

\begin{frame}
\frametitle{Sistemas de Provas}
\begin{itemize}
    \item Sistemas/C�lculos de provas servem para construir provas de uma maneira muito formal.
    \item S�o uma cole��o de regras que explicam como derivar novas f�rmulas.
    \item Um sistema de prova pode ser utilizado na formaliza��o de diversas l�gicas,
      como L�gica Proposicional e L�gica de Predicados.
    \item Os principais sistemas de provas s�o a Dedu��o Natural e o C�lculo de Hilbert.
\end{itemize}
\end{frame}

\begin{frame}[fragile]
\frametitle{Fragmento Proposicional Cl�ssico da Dedu��o Natural}
\begin{center}
        \begin{prooftree}
            \AxiomC{}
            \RightLabel{$(\small{hip)}$}
            \UnaryInfC{$u : A$}
        \end{prooftree}
\end{center}

\begin{center}
\begin{minipage}{.4\textwidth}
        \begin{prooftree}
            \AxiomC{$A \rightarrow B$}
            \AxiomC{$A$}
            \RightLabel{$(\rightarrow^-)$}
            \BinaryInfC{$B$}
        \end{prooftree}
\end{minipage}
\begin{minipage}{.4\textwidth}
        \begin{prooftree}
            \AxiomC{$B$}
            \RightLabel{$(\rightarrow^+) \: [u : A]$}
            \UnaryInfC{$A \rightarrow B$}
        \end{prooftree}
\end{minipage}
\end{center}
        
\begin{center}
\begin{minipage}{.3\textwidth}
        \begin{prooftree}
            \AxiomC{$A \wedge B$}
            \RightLabel{$(\wedge_l^-)$}
            \UnaryInfC{$A$}
       \end{prooftree}
\end{minipage}
\begin{minipage}{.3\textwidth}
        \begin{prooftree}
            \AxiomC{$  A \wedge B$}
            \RightLabel{$(\wedge_r^-)$}
            \UnaryInfC{$B$}
       \end{prooftree}
\end{minipage}
\begin{minipage}{.3\textwidth}
        \begin{prooftree}
            \AxiomC{$A$}
            \AxiomC{$B$}
            \RightLabel{$(\wedge^+)$}
            \BinaryInfC{$A \wedge B$}
       \end{prooftree}
\end{minipage}
\end{center}

\begin{center}
\begin{minipage}{.5\textwidth}
        \begin{prooftree}
            \AxiomC{$A \vee B$}
            \AxiomC{$A \rightarrow C$}
            \AxiomC{$B \rightarrow C$}
            \RightLabel{$(\vee^-)$}
            \TrinaryInfC{$C$}
       \end{prooftree}
\end{minipage}
\end{center}

\begin{center}
\begin{minipage}{.3\textwidth}
        \begin{prooftree}
            \AxiomC{$A$}
            \RightLabel{$(\vee_l^+)$}
            \UnaryInfC{$A \vee B$}
       \end{prooftree}
\end{minipage}
\begin{minipage}{.2\textwidth}
        \begin{prooftree}
            \AxiomC{$B$}
            \RightLabel{$(\vee_r^+)$}
            \UnaryInfC{$A \vee B$}
       \end{prooftree}
\end{minipage}
\end{center}

\begin{center}
\begin{minipage}{.3\textwidth}
        \begin{prooftree}
            \AxiomC{$\bot$}
            \RightLabel{$(efq)$}
            \UnaryInfC{$A$}
        \end{prooftree}
\end{minipage}
\begin{minipage}{.3\textwidth}
        \begin{prooftree}
            \AxiomC{$\neg \neg A$}
            \RightLabel{$(raa)$}
            \UnaryInfC{$A$}
        \end{prooftree}
\end{minipage}
\end{center}
onde $\neg A \Rightarrow A \rightarrow \bot$.
\end{frame}

\begin{frame}
\frametitle{Exemplo de Prova em DN}
        \begin{prooftree}
            \AxiomC{$$}
            \RightLabel{(\small{hip)}}
            \UnaryInfC{$u : (A \vee \neg A) \rightarrow \bot$}
            \AxiomC{$$}
            \RightLabel{(\small{hip)}}
            \UnaryInfC{$u : (A \vee \neg A) \rightarrow \bot$}
            \AxiomC{$$}
            \RightLabel{(\small{hip)}}
            \UnaryInfC{$v : A$}
            \RightLabel{$(\vee^+_r)$}
            \UnaryInfC{$A \vee \neg A$}
            \BinaryInfC{$\bot$}
            \RightLabel{$(\rightarrow^+) [v]$}
            \UnaryInfC{$A \rightarrow \bot$}
            \RightLabel{$(\vee^+_l)$}
            \UnaryInfC{$(A \vee \neg A)$}
            \RightLabel{$(\rightarrow^-)$}
            \BinaryInfC{$\bot$}
            \RightLabel{$(\rightarrow^+) [u]$}
            \UnaryInfC{$((A \vee \neg A) \rightarrow \bot) \rightarrow \bot$}
            \RightLabel{$(raa)$}
            \UnaryInfC{$A \vee \neg A$}
        \end{prooftree}
 
\end{frame}

\begin{frame}
\frametitle{C�lculo Lambda}
\begin{itemize}
  \item O C�lculo Lambda � um modelo de computa��o criado por Alonzo Church em 1933.
  \item O proposito inicial do C�lculo Lambda foi ser uma linguagem de macros para um l�gica e assim demonstrar a indecibilidade do problema da parada.
  \item Funciona como um sistema de reescrita: existem regras para reescrever express�es.
  \item A primeira linguagem de programa��o. Uma d�cada antes do primeiro computador.
\end{itemize}
Sintaxe:
\begin{align*}
  e \: := (e \: e') \: | \: (\lambda x.e) \: | \: x
\end{align*}
Reescrita:
\begin{align*}
  (\lambda x. e) \: e' \Rightarrow_\beta [e'/x]e
\end{align*}
\end{frame}

\begin{frame}
\frametitle{Exemplos de Computa��o em CL}
1)
\begin{align*}
  (\lambda x.x) \: y \Rightarrow_\beta y
\end{align*}
2)
\begin{align*}
  (\lambda x.x) \: (\lambda x.x) \Rightarrow_\beta \lambda x.x
\end{align*}
3)
\begin{align*}
  ((\lambda x.\lambda y . y) \: a) b \Rightarrow_\beta b
\end{align*}
4)
\begin{align*}
  (\lambda x.x x) \: (\lambda x.x x) \Rightarrow_\beta (\lambda x.x x) \: (\lambda x.x x)
\end{align*}
\end{frame}

\begin{frame}
\frametitle{C�lculo Lambda Tipado}
O C�lculo Lambda � muito poderoso, permite criar o Rightarrowalente de f�rmulas l�gicas infinitas e assim
a l�gica representada � inconsistente.

Para evitar paradoxos Church utilizou o mesmo truque que Bertrand Russel: Type Thoery.

\end{frame}

\begin{frame}
\frametitle{C�lculo Lambda Tipado}
\begin{center}
        \begin{prooftree}
            \AxiomC{}
            \RightLabel{$(\small{hip)}$}
            \UnaryInfC{$x:A \vdash x: A$}
        \end{prooftree}
\end{center}

\begin{center}
\begin{minipage}{.6\textwidth}
        \begin{prooftree}
            \AxiomC{$\Gamma \vdash e : A \rightarrow B$}
            \AxiomC{$\Gamma \vdash e' : A$}
            \RightLabel{$(\rightarrow^-)$}
            \BinaryInfC{$e \: e' : \Gamma \vdash B$}
        \end{prooftree}
\end{minipage}
\begin{minipage}{.3\textwidth}
        \begin{prooftree}
            \AxiomC{$\Gamma, x:A \vdash e:B$}
            \RightLabel{$(\rightarrow^+)$}
            \UnaryInfC{$\Gamma \vdash \lambda x. e : A \rightarrow B$}
        \end{prooftree}
\end{minipage}
\end{center}
        
\begin{center}
\begin{minipage}{.3\textwidth}
        \begin{prooftree}
            \AxiomC{$\Gamma \vdash e : A * B$}
            \RightLabel{$(fst)$}
            \UnaryInfC{$\Gamma \vdash fst \: e : A$}
       \end{prooftree}
\end{minipage}
\begin{minipage}{.3\textwidth}
        \begin{prooftree}
            \AxiomC{$\Gamma \vdash e :  A * B$}
            \RightLabel{$(snd)$}
            \UnaryInfC{$\Gamma \vdash snd \: e : B$}
       \end{prooftree}
\end{minipage}

\begin{minipage}{.3\textwidth}
        \begin{prooftree}
            \AxiomC{$\Gamma \vdash e : A$}
            \AxiomC{$\Gamma \vdash e' : B$}
            \RightLabel{$(pair)$}
            \BinaryInfC{$\Gamma \vdash (e, e') : A \wedge B$}
       \end{prooftree}
\end{minipage}
\end{center}
\begin{align*}
  fst (e, e') & \Rightarrow_\beta e \\
  snd (e, e') & \Rightarrow_\beta e'
\end{align*}
\end{frame}

\begin{frame}
\frametitle{C�lculo Lambda Tipado}
\begin{center}
        \begin{prooftree}
            \AxiomC{$\Gamma \vdash e : A + B$}
            \AxiomC{$\Gamma \vdash e' : A \rightarrow C$}
            \AxiomC{$\Gamma \vdash e'' : B \rightarrow C$}
            \RightLabel{$(+^-)$}
            \TrinaryInfC{$\Gamma \vdash case (e, e', e'') : C$}
       \end{prooftree}
\end{center}

\begin{center}
\begin{minipage}{.4\textwidth}
        \begin{prooftree}
            \AxiomC{$\Gamma \vdash e : A$}
            \RightLabel{$+^+_l$}
            \UnaryInfC{$\Gamma \vdash (0, e) : A + B$}
       \end{prooftree}
\end{minipage}
\begin{minipage}{.4\textwidth}
        \begin{prooftree}
            \AxiomC{$B$}
            \RightLabel{$+^+_r$}
            \UnaryInfC{$\Gamma \vdash (1, e) : A + B$}
       \end{prooftree}
\end{minipage}
\end{center}

\begin{align*}
  case((0, e), e', e'') & \Rightarrow_\beta e' \: e \\
  case((1, e), e', e'') & \Rightarrow_\beta e'' \: e
\end{align*}
\end{frame}

\section{Extra��o de Programas}

\begin{frame}
\frametitle{Realizability}
\end{frame}

\section{Extra��o em Minlog}

\begin{frame}
\frametitle{Realizability}
\end{frame}


\end{document}
%%% Local Variables:
%%% mode: latex
%%% TeX-master: t
%%% End:
