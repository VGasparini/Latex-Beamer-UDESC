
\documentclass{beamer}

\usepackage{graphicx,hyperref,udesc,url}
\usepackage[latin1]{inputenc}
%\usepackage[T1]{fontenc}
\usepackage{booktabs}
\usepackage[portuges]{babel}


\title[Uso de Coq para verificação de propriedades de sistemas de tipos]{Uso de Coq para verificação de propriedades de sistemas de tipos}

\author[Rafael Castro]{
    Rafael Castro\\\medskip
    {\small \url{rafaelcgs10@gmail.com}}}

\date{22 de Junho de 2018}

    \institute[UDESC]{
        Departamento de Ci\^encia da Computa\c{c}\~ao \\
            Centro de Ci\^encias e Tecnol\'ogicas\\
            Universidade do Estado de Santa Catarina}

\begin{document}

\begin{frame}
\titlepage

\end{frame}

\section{Introdução}
\begin{frame}
\frametitle{O que são assistentes de provas?}
\begin{itemize}
    \item Assistentes de provas ou provadores interativos são programas para o desenvolvimento de provas
          formais.
    \item O núcleo de um assistente de provas é um verificador, que verifica a consistência lógica da prova.
    \item Fornecem de maneira interativa de visualizar as informações sobre o estado atual da prova.
    \item A verficação humana de provas é demorada e sujeita a falhas: Último Teorema de Fermat.

      Assistentes de provas permitem provar coisas que não seriam realizáveis somente com papel e caneta
\end{itemize}
\end{frame}

\section{Coq}
\begin{frame}
\frametitle{O que é Coq?}
\begin{itemize}
    \item Coq é um assistentes de provas (dã) desenvolvido desde 1984 pelo French Institute for Research in Computer Science and Automation (INRIA).
    \item Coq é fruto de sistemas de tipos: baseado em \textbf{Higher order dependently typed polymorphic lambda calculus}, o nomeado Calculus of Constructions (CoC).
    \item Coq suporta programação com tipos dependentes.
    \item Coq tem várias linguagens:
        \begin{enumerate}
            \item Gallina: linguagem de programação funcional total e com tipos dependentes.
            \item Vernecular: linguagem de comandos. Enunciar teoremas, funções…
            \item Tatic: linguagem para criar provas com a ajuda do assistente/verificador.
            \item LTac: criação de novas táticas e procedimentos de prova.
        \end{enumerate}
\end{itemize}
\end{frame}

\section{A linguagem}
\end{document}


